\documentclass[usepdftitle=false,24pt]{beamer}
\usepackage[utf8]{inputenc}
\usepackage[T1]{fontenc}
\usepackage{carlito}
\usepackage{graphicx}
\usepackage{hyperref}
\usepackage{listings}
\usepackage{xcolor}
\usepackage{subcaption}
\usepackage{movie15}

\setbeamertemplate{bibliography item}[text]

\setbeamertemplate{section in toc}{\Large{\inserttocsectionnumber.}~\inserttocsection}

\setbeamertemplate{caption}[numbered]

\renewcommand{\figurename}{Rys.}

\usetheme[vertical, pagenumbers, hr=false, lang=pl]{NewPwr}

\setbeamersize{text margin left=4mm,text margin right=4mm}

\def\mytitle{Metody Sztucznej Inteligencji}
\def\myauthor{Dawid Łukasiewicz, 252891}
\def\addinfo{Model przewidywania pogody}
\def\mydate{25.01.2023}
\date{}


\begin{document}

\begin{frame}
    \begin{titlepage}
        \centering
        {\huge\bfseries \mytitle\\}
        \vspace{2cm}
        {\Large \addinfo\\}
        \vspace{.4cm}
        {\Large \myauthor\\}
        \vspace{1cm}
        {\large Wrocław \mydate }
        \vfill

    \end{titlepage}
\end{frame}

\begin{frame}
    \frametitle{Wstęp}

    Celem modelu było przewidywanie pogody na następne 7 dni na podstawie poprzednich 30 dni.

    \vspace*{1cm}

    Pod uwagę brano tylko 3 cechy każdego dnia:
    \begin{itemize}
            \item opady deszczu,
            \item maksymalna temperatura,
            \item minimalna temperatura.
        \end{itemize}

    \vspace*{1cm}

    Na podstawie wartości 3 cech z 30 dni, model szacował wartości tych  samych cech dla kolejnych 7 dni.

\end{frame}


% \section{Bibliografia}
\begin{frame}[allowframebreaks]
    \frametitle{Bibliografia}
    % \bibliographystyle{ieeetr}
    % \bibliography{biblio.bib}

\end{frame}



\end{document}